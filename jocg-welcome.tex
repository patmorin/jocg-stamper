\documentclass{jocg}
\usepackage{amsfonts}

% NOTE - the title is UPPERCASE
\title{
  \MakeUppercase{Welcome from the Editors-in-Chief}
}

\author{Kenneth L. Clarkson%
        \thanks{\affil{IBM Almaden}}\,
        and G\"unter Rote%
        \thanks{\affil{Freie Universit\"at Berlin}}}

% NOTE - author names are comma-separated, and extra space is added after
% footnote markers
%\author{%
%  Peter~Hampson,%
%  \thanks{\affil{Oxford University},
%          \email{pete@oxford.com.au.gov.org}}\,
%  Patrick~Morin,%
%  \thanks{\affil{Carleton University},
%          \email{\{morin,wolfie\}@scs.carleton.ca}}\,
%  Joachim~Gudmundsson,%
%  \thanks{\affil{NICTA},
%          \email{joachim.gudmundsson@nicta.com.au}}\,
%  and Wolfgang~Snicklefritz\footnotemark[3]
%}

% The following are just some examples of how to use the amsthm
% environments, but will probably be used in most submissions
\usepackage{amsthm}
\usepackage{hyperref}
\theoremstyle{plain}
\newtheorem{theorem}{Theorem}
\theoremstyle{definition}
\newtheorem{question}{Question}

%\renewcommand\baselinestretch{0.98}
%\setlength{\parskip}{.4cm}

\begin{document}
\maketitle

We are pleased to welcome readers to this inaugural issue of the
\textit{Journal of Computational Geometry}.

\textit{Computational geometry} is, broadly, the study of algorithmic problems in a
geometric setting. While such questions can be traced back to antiquity, the modern
study, as a branch of theoretical computer science, has its origins in the
nineteen-seventies, and thrives to this day. The computational geometry community recently
\href{http://www.madalgo.au.dk/socg2009/Html/Sattelite Events/25_Anniversary.html}{celebrated}
the occasion of the twenty-fifth \href{http://www.madalgo.au.dk/socg2009/index.html}{Annual Symposium on Computational Geometry},
held in Aarhus, Denmark, and there are regular meetings devoted to the field in 
Europe, Canada, Japan, and the United States.\footnote{%
\href{http://www.eurocg.org/}{The European Workshop on Computational Geometry};
\href{http://www.cccg.ca/}{the Canadian Conference on Computational Geometry};
\href{http://www.jaist.ac.jp/~uehara/JCCGG09/}{the Japan Conference on Computational Geometry and Graphs};
\href{http://www.cs.tufts.edu/research/geometry/FWCG09/}{the Fall Workshop on Computational Geometry}
} Computational geometry is also presented at more general conferences
in theoretical computer science; moreover, since ``geometry is everywhere in
nature,''\footnote{Auguste Rodin (1840--1917)} the beauty and utility of the geometrical
viewpoint arises naturally also in more applied areas such as geographical
information systems, graph drawing, graphics, and geometric design and
manufacturing. While keeping to strict standards of quality, this journal
will strive for inclusiveness and breadth.

There are already three journals devoted to computational geometry:
\textit{Discrete and Computational Geometry},
\textit{Computational Geometry: Theory and Applications},
and the
\textit{International Journal of Computational Geometry and Applications}.
Why, then, a new journal? These journals are produced and distributed
by commercial publishers in the traditional hardcopy form.
In contrast, the \textit{Journal of Computational Geometry} is
online only, and is free, in multiple ways: it is available
without restriction to readers
at no cost to authors and readers, and authors retain the rights to their works,
protected by a Creative Commons license. The Journal exists solely to serve
the needs of the research community and the public. In addition to simple
convenience, the free and online format allows the Journal to publish only as
high-quality submissions are received, and online distribution speeds
publication. Also, there is the possibility of archiving supplementary
material, such as code and data, with the published articles.

While the Journal is free, it depends very much on the efforts of its authors,
reviewers, and editors. The Journal was founded at the initiative of Joachim
Gudmundsson and Pat Morin, who are acting as managing editors. Their initial
editorial team comprises an international mix of younger as well as
established researchers, working in both theory and applications, with
strong research records in areas including such diverse topics as metric embeddings,
origami, airline scheduling, sensor networks, compressed sensing, probabilistic
roadmaps, geometric discrepancy, algorithms for massive datasets, and persistent homology.

We have every hope that the Journal will strongly support the vibrant field
of computational geometry.

\noindent Kenneth L. Clarkson and G\"{u}nter Rote\\
\noindent Editors-in-Chief


\end{document}
